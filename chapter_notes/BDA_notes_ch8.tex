\documentclass[a4paper,11pt,english]{article}

\usepackage{babel}
\usepackage[latin1]{inputenc}
\usepackage[T1]{fontenc}
% \usepackage[T1,mtbold,lucidacal,mtplusscr,subscriptcorrection]{mathtime}
\usepackage{times}
\usepackage{amsmath}
\usepackage{microtype}
\usepackage{url}
\urlstyle{same}

\usepackage[bookmarks=false]{hyperref}
\hypersetup{%
  bookmarksopen=true,
  bookmarksnumbered=true,
  pdftitle={Bayesian data analysis},
  pdfsubject={Comments},
  pdfauthor={Aki Vehtari},
  pdfkeywords={Bayesian probability theory, Bayesian inference, Bayesian data analysis},
  pdfstartview={FitH -32768}
}


% if not draft, smaller printable area makes the paper more readable
\topmargin -4mm
\oddsidemargin 0mm
\textheight 225mm
\textwidth 160mm

%\parskip=\baselineskip
\def\eff{\mathrm{rep}}

\DeclareMathOperator{\E}{E}
\DeclareMathOperator{\Var}{Var}
\DeclareMathOperator{\var}{var}
\DeclareMathOperator{\Sd}{Sd}
\DeclareMathOperator{\sd}{sd}
\DeclareMathOperator{\Bin}{Bin}
\DeclareMathOperator{\Beta}{Beta}
\DeclareMathOperator{\Invchi2}{Inv-\chi^2}
\DeclareMathOperator{\NInvchi2}{N-Inv-\chi^2}
\DeclareMathOperator{\logit}{logit}
\DeclareMathOperator{\N}{N}
\DeclareMathOperator{\U}{U}
\DeclareMathOperator{\tr}{tr}
%\DeclareMathOperator{\Pr}{Pr}
\DeclareMathOperator{\trace}{trace}
\DeclareMathOperator{\rep}{\mathrm{rep}}

\pagestyle{empty}

\begin{document}
\thispagestyle{empty}

\section*{Bayesian data analysis -- reading instructions 8} 
\smallskip
{\bf Aki Vehtari}
\smallskip

\subsection*{Chapter 8}

In the earlier chapters it was assumed that the data collection is
ignorable. Chapter 8 explains when data collection can be ignorable
and when we need to model also the data collection.
We don't have time to go through chapter 8 in BDA course at Aalto, but
it is highly recommended that you would read it in the end or after
the course. Most important parts are 8.1, 8.5, pp 220--222 of 8.6, and
8.8, and you can get back to the other sections later.

Outline of the chapter 8 (* denotes the most important parts)
\begin{list}{$\bullet$}{\parsep=0pt\itemsep=2pt}
\item 8.1 Bayesian inference requires a model for data collection  (*)
\item 8.2 Data-collection models and ignorability
\item 8.3 Sample surveys
\item 8.4 Designed experiments
\item 8.5 Sensitivity and the role of randomization (*)
\item 8.6 Observational studies (* pp 220--222)
\item 8.7 Censoring and truncation (*)
\end{list}

Most important terms in the chapter
\begin{list}{$\bullet$}{\parsep=0pt\itemsep=2pt}
\item observed data
\item complete data
\item missing data
\item stability assumption
\item data model
\item inclusion model
\item complete data likelihood
\item observed data likelihood
\item finite-population and superpopulation inference
\item ignorability
\item ignorable designs
\item propensity score
\item sample surveys
\item random sampling of a finite population
\item stratified sampling
\item cluster sampling
\item designed experiments
\item complete randomization
\item randomized blocks and latin squares
\item sequntial designs
\item randomization given covariates
\item observational studies
\item censoring
\item truncation
\item missing completely at random
\end{list}

% Gelman: ``All contexts where the model is fit to data that are not
% necessarily representative of the population that is the target of
% study. The key idea is to include in the Bayesian model an inclusion
% variable with a probability distribution that represents the process
% by which data become observed.''

\end{document}


%%% Local Variables: 
%%% TeX-PDF-mode: t
%%% TeX-master: t
%%% End: 
